\documentclass[12pt, a4paper]{article}
\usepackage[utf8]{inputenc}
\usepackage[spanish]{babel}
\usepackage{graphicx}
\usepackage{amsmath}
\usepackage{amssymb}
\usepackage{float}
\usepackage{hyperref}
\usepackage{geometry}
\usepackage{titlesec}
\usepackage{booktabs}
\usepackage{listings}
\usepackage{xcolor}
\usepackage{caption}
\usepackage{subcaption}
\usepackage{cite}
\usepackage{setspace}
\usepackage{tocloft}
\usepackage{multirow}
\usepackage{array}

\geometry{left=2.5cm, right=2.5cm, top=2.5cm, bottom=2.5cm}
\onehalfspacing

% Configuración del índice
\renewcommand{\cftsecleader}{\cftdotfill{\cftdotsep}}
\setcounter{tocdepth}{3}

% Configuración de código
\definecolor{codegreen}{rgb}{0,0.6,0}
\definecolor{codegray}{rgb}{0.5,0.5,0.5}
\definecolor{codepurple}{rgb}{0.58,0,0.82}
\definecolor{backcolour}{rgb}{0.95,0.95,0.92}

\lstdefinestyle{mystyle}{
    backgroundcolor=\color{backcolour},   
    commentstyle=\color{codegreen},
    keywordstyle=\color{magenta},
    numberstyle=\tiny\color{codegray},
    stringstyle=\color{codepurple},
    basicstyle=\ttfamily\footnotesize,
    breakatwhitespace=false,         
    breaklines=true,                 
    captionpos=b,                    
    keepspaces=true,                 
    numbers=left,                    
    numbersep=5pt,                  
    showspaces=false,                
    showstringspaces=false,
    showtabs=false,                  
    tabsize=2
}

\lstset{style=mystyle}

% Configuración de la portada
\title{
    \vspace{2cm}
    \textbf{\LARGE Restauración y Recuperación de Información en Imágenes Digitales: Un Enfoque desde la Teoría de la Información y el Procesamiento de Señales}\\
    \vspace{1cm}
    \large Proyecto Final de Investigación\\
    \vspace{0.5cm}
    \large Curso: Teoría de la Información y Sistemas de Comunicación
}
\author{
    \textbf{Estudiante: [Nombre del Estudiante]}\\
    \vspace{0.5cm}
    Docente: [Nombre del Docente]\\
    \vspace{0.5cm}
    Programa Académico: Ingeniería de Sistemas / Electrónica\\
    Universidad Nacional de Colombia\\
    Facultad de Ingeniería\\
    Departamento de Ingeniería de Sistemas e Industrial
}
\date{\today}

\begin{document}

% 1. Portada
\begin{titlepage}
    \centering
    \vspace*{1cm}
    {\LARGE \textbf{Universidad Nacional de Colombia}}\\[0.5cm]
    {\Large Facultad de Ingeniería}\\[0.5cm]
    {\Large Departamento de Ingeniería de Sistemas e Industrial}\\[2cm]
    
    {\Huge \textbf{Restauración de Imágenes Digital}}\\[0.5cm]
    {\Large Aplicación de Teoría de la Información y Procesamiento de Señales}\\[2cm]
    
    \textbf{Autores:}\\
    [Alejandro Argüello Muñoz]\\
    Juan Luis Vergara Novoa\\[1cm]
    
    \textbf{Docente:}\\
    [Oswaldo Rojas Camacho]\\[2cm]
    
    \vfill
    
    \textbf{Bogotá D.C., Colombia}\\
    \today
    \thispagestyle{empty}
\end{titlepage}

% Índice General
\tableofcontents
\newpage

% 2. Resumen / Abstract
\section{Resumen}
Este proyecto aborda el problema fundamental de la degradación de información en sistemas de comunicación visual, donde el ruido y las distorsiones del canal reducen la entropía útil de las imágenes. Fundamentado en la Teoría de la Información de Shannon, se analiza la imagen como una fuente de información estocástica y el proceso de degradación como un canal ruidoso con capacidad limitada. Se desarrolló una plataforma de software integral que implementa técnicas de procesamiento de señales en el dominio de la frecuencia (Transformada de Fourier) para combatir el ruido periódico, y métodos espaciales especializados (Filtro Gaussiano, Mediana Adaptativa) para maximizar la relación señal-ruido (SNR) en presencia de ruido Gaussiano e impulsivo. Además, se integra el algoritmo estado del arte BM3D para la restauración avanzada de ruido Gaussiano. Los resultados demuestran que la redundancia espacial y espectral es la clave para la restauración exitosa, logrando mejoras significativas en métricas objetivas como PSNR y SSIM.

\textbf{Palabras clave:} Teoría de la Información, Entropía, Capacidad de Canal, Transformada de Fourier, Ruido AWGN, Denoising, Deep Learning, PSNR.

% 3. Introducción
\section{Introducción}
La transmisión de imágenes digitales es omnipresente en la ingeniería moderna, desde redes IoT de vigilancia hasta sistemas de teledetección satelital. Sin embargo, todo canal físico introduce incertidumbre. El ruido térmico en los sensores, la interferencia electromagnética en los enlaces de radio y los errores de cuantización actúan como fuentes de entropía negativa que degradan la calidad del mensaje visual.

Este proyecto se justifica en la necesidad de desarrollar sistemas robustos capaces de recuperar la información original $X$ a partir de una observación ruidosa $Y$. En el contexto de la Teoría de la Información, esto equivale a maximizar la Información Mutua $I(X;Y)$ mediante el post-procesamiento. El documento estructura una solución ingenieril completa: desde el modelado matemático del problema y la simulación de canales ruidosos, hasta la implementación y validación crítica de algoritmos de restauración espectrales y espaciales.

% 4. Planteamiento del Problema
\section{Planteamiento del Problema}

\subsection{Descripción General del Problema}
En sistemas modernos de comunicación visual, la calidad de las imágenes transmitidas se ve comprometida por múltiples fuentes de degradación que operan tanto en el dominio espacial como en el dominio de frecuencia. Estas degradaciones no solo afectan la calidad estética de la imagen, sino que reducen significativamente la cantidad de información útil que puede ser extraída por sistemas de visión artificial, análisis médico, o inspección industrial.

\subsection{Formulación Matemática del Problema de Restauración}
El problema de restauración de imágenes puede formularse matemáticamente como un problema inverso mal condicionado (ill-posed problem). Dada una imagen observada $g(x,y)$ que ha sufrido degradación, se busca estimar la imagen original $f(x,y)$. El proceso de degradación se modela típicamente como:

\begin{equation}
    g(x,y) = h(x,y) * f(x,y) + \eta(x,y)
\end{equation}

donde:
\begin{itemize}
    \item $g(x,y)$ es la imagen degradada observada
    \item $f(x,y)$ es la imagen original (desconocida)
    \item $h(x,y)$ es la función de degradación (blur kernel, función de transferencia óptica)
    \item $*$ denota la operación de convolución
    \item $\eta(x,y)$ es el ruido aditivo
\end{itemize}

\subsection{Descomposición del Problema en Subproblemas}
El problema general de restauración puede descomponerse en tres subproblemas fundamentales:

\begin{enumerate}
    \item \textbf{Identificación del Tipo de Ruido}: Diferentes tipos de ruido requieren estrategias de restauración específicas. Un diagnóstico incorrecto conduce a algoritmos ineficaces.
    \item \textbf{Estimación de Parámetros de Degradación}: Para ruido Gaussiano, estimar $\sigma$; para desenfoque, estimar el kernel $h(x,y)$; para ruido periódico, identificar frecuencias dominantes.
    \item \textbf{Selección y Aplicación del Algoritmo Óptimo}: Elegir entre métodos espaciales, espectrales o basados en aprendizaje profundo según el tipo y severidad de la degradación.
\end{enumerate}

\subsection{Problemas Específicos Abordados}
\begin{enumerate}
    \item \textbf{Ruido Gaussiano Aditivo (AWGN)}: Modelo de ruido más común en sistemas de comunicación, que sigue una distribución normal $\mathcal{N}(0, \sigma^2)$. El desafío radica en separar señal de ruido sin eliminar detalles finos.
    
    \item \textbf{Ruido Sal y Pimienta}: Ruido impulsivo que corrompe píxeles individuales con valores extremos. Presenta un reto particular para algoritmos lineales, ya que contamina menos del 10\% de los píxeles pero con alta intensidad.
    
    \item \textbf{Ruido Periódico}: Interferencia estructurada que aparece como patrones repetitivos en la imagen. Este tipo de ruido es especialmente problemático porque no es estacionario y tiene características espectrales bien definidas.
    
    \item \textbf{Compromiso (Trade-off) entre Eliminación de Ruido y Preservación de Bordes}: Problema fundamental donde la reducción agresiva de ruido tiende a suavizar bordes y texturas importantes.
\end{enumerate}

\subsection{Marco de Evaluación del Problema}
La complejidad del problema se evalúa mediante:
\begin{itemize}
    \item \textbf{Relación Señal-Ruido (SNR)}: Diferencia en dB entre la potencia de señal y la potencia de ruido.
    \item \textbf{Entropía Relativa}: Aumento en la incertidumbre (entropía) causada por el ruido.
    \item \textbf{Capacidad de Canal Degradado}: Reducción en la capacidad teórica del canal imagen según Shannon-Hartley.
\end{itemize}

\subsection{Justificación e Importancia}
La solución a este problema tiene aplicaciones en múltiples dominios:
\begin{itemize}
    \item \textbf{Medicina}: Mejora de imágenes de resonancia magnética, tomografías, y microscopía.
    \item \textbf{Vigilancia y Seguridad}: Clarificación de imágenes de cámaras de seguridad en condiciones adversas.
    \item \textbf{Teledetección}: Procesamiento de imágenes satelitales y aéreas afectadas por condiciones atmosféricas.
    \item \textbf{Sistemas de Comunicación}: Recuperación de imágenes transmitidas por canales ruidosos.
\end{itemize}

La solución propuesta se fundamenta en principios de Teoría de la Información, buscando maximizar la información mutua $I(X;Y)$ entre la imagen original $X$ y la imagen restaurada $Y$, sujeto a las limitaciones impuestas por el teorema de Shannon-Hartley.

% 5. Objetivos
\section{Objetivos}

\subsection{Objetivo General}
Desarrollar una plataforma de software para la restauración de imágenes digitales que aplique conceptos de Teoría de la Información y Procesamiento de Señales para evaluar y comparar la eficacia de algoritmos espectrales, espaciales y de aprendizaje profundo en la recuperación de información visual.

\subsection{Objetivos Específicos}
\begin{enumerate}
    \item \textbf{Analizar} la señal de imagen mediante métricas de información (Entropía, SNR) para caracterizar el impacto de diferentes tipos de ruido.
    \item \textbf{Implementar} la Transformada de Fourier 2D para identificar y filtrar componentes de ruido periódico, demostrando la utilidad del análisis espectral.
    \item \textbf{Desarrollar} algoritmos de filtrado espacial (Gaussiano, Mediana Adaptativa) y el algoritmo BM3D, evaluando su desempeño en términos de preservación de bordes y reducción de entropía de ruido.
    \item \textbf{Evaluar} el desempeño del sistema mediante métricas objetivas (PSNR, SSIM) y subjetivas, determinando el método óptimo para cada escenario de canal.
\end{enumerate}

% 6. Marco Teórico
\section{Marco Teórico}

\subsection{Modelo Matemático de Degradación de Imágenes}
El proceso de degradación en sistemas de comunicación visual se modela matemáticamente como un sistema lineal invariante en el espacio (LSI) combinado con ruido aditivo:

\begin{equation}
    g(x,y) = \iint_{-\infty}^{\infty} f(\alpha,\beta)h(x-\alpha, y-\beta)d\alpha d\beta + \eta(x,y)
\end{equation}

En el caso discreto, esta ecuación se convierte en una convolución 2D:

\begin{equation}
    g[i,j] = \sum_{m=-M}^{M}\sum_{n=-N}^{N} f[i-m, j-n]h[m,n] + \eta[i,j]
\end{equation}

La naturaleza mal condicionada del problema inverso (deconvolución) se debe a que la transformada de Fourier de $h(x,y)$, denotada $H(u,v)$, puede tener ceros o valores cercanos a cero, haciendo la inversión numéricamente inestable.

\subsection{Teoría de la Información Aplicada a Imágenes}

\subsubsection{Entropía de Shannon para Imágenes}
La entropía mide la cantidad promedio de información contenida en una imagen. Para una imagen digital con $L$ niveles de gris y distribución de probabilidad $p(r_k)$:

\begin{equation}
    H = -\sum_{k=0}^{L-1} p(r_k) \log_2 p(r_k) \quad \text{[bits/píxel]}
\end{equation}

El ruido aumenta la entropía de la imagen porque distribuye la intensidad de los píxeles de manera más uniforme. Una imagen sin ruido tiene histogramas concentrados (baja entropía en regiones uniformes), mientras que el ruido dispersa estos valores.

\subsubsection{Información Mutua y Capacidad de Canal}
La información mutua $I(X;Y)$ entre la imagen original $X$ y la observada $Y$ mide cuánta información sobre $X$ se preserva en $Y$:

\begin{equation}
    I(X;Y) = H(Y) - H(Y|X)
\end{equation}

Para un canal AWGN, el teorema de Shannon-Hartley establece la capacidad máxima:

\begin{equation}
    C = B \log_2\left(1 + \frac{S}{N}\right) \quad \text{[bits/segundo]}
\end{equation}

donde $B$ es el ancho de banda y $S/N$ es la relación señal-ruido. En procesamiento de imágenes, podemos interpretar $B$ como la máxima frecuencia espacial que puede ser transmitida.

\subsection{Tipos de Ruido y sus Características}

\subsubsection{Ruido Gaussiano Aditivo (AWGN)}
\begin{itemize}
    \item \textbf{Distribución}: $\eta \sim \mathcal{N}(0, \sigma^2)$
    \item \textbf{Función de Densidad de Probabilidad}: 
    \begin{equation}
        p(\eta) = \frac{1}{\sigma\sqrt{2\pi}} e^{-\frac{\eta^2}{2\sigma^2}}
    \end{equation}
    \item \textbf{Espectro de Potencia}: Constante en todas las frecuencias (blanco)
    \item \textbf{Autocorrelación}: $R_{\eta\eta}(\tau) = \sigma^2 \delta(\tau)$
    \item \textbf{Aplicaciones}: Modela ruido térmico en sensores, ruido de cuantización, y efectos atmosféricos.
\end{itemize}

\subsubsection{Ruido Sal y Pimienta (Impulsivo)}
\begin{itemize}
    \item \textbf{Modelo Matemático}:
    \begin{equation}
        \eta(x,y) = \begin{cases}
            a & \text{con probabilidad } p_a \\
            b & \text{con probabilidad } p_b \\
            0 & \text{con probabilidad } 1-p_a-p_b
        \end{cases}
    \end{equation}
    donde típicamente $a=0$ (pimienta) y $b=255$ (sal).
    \item \textbf{Características}: No estacionario, no Gaussiano, con valores extremos.
    \item \textbf{Detección}: Análisis de discontinuidades locales y valores saturados.
\end{itemize}

\subsubsection{Ruido Periódico}
\begin{itemize}
    \item \textbf{Modelo}: $\eta(x,y) = A \sin(2\pi(u_0 x + v_0 y) + \phi)$
    \item \textbf{Espectro de Fourier}: Picos delta en las frecuencias $(u_0, v_0)$ y sus simétricas.
    \item \textbf{Autocorrelación}: Función periódica con la misma frecuencia espacial.
    \item \textbf{Fuentes}: Interferencia de líneas eléctricas (50/60 Hz), aliasing en digitalización, patrones de moiré.
\end{itemize}

\subsection{Métodos de Restauración}

\subsubsection{Métodos en el Dominio de Frecuencia}

\paragraph{Transformada de Fourier 2D}
La transformada de Fourier descompone la imagen en sus componentes de frecuencia espacial:
\end{equation}

donde $D_k(u,v) = \sqrt{(u - u_k)^2 + (v - v_k)^2}$ es la distancia al k-ésimo centro de notch.

\subsubsection{Métodos en el Dominio Espacial}

\paragraph{Filtro de Mediana Adaptativa}
Algoritmo no lineal que preserva bordes mejor que filtros lineales:

\begin{algorithmic}
\State Inicializar ventana $W$ de tamaño $w_{\min} \times w_{\min}$
\Repeat
    \State Calcular $z_{\min}, z_{\max}, z_{\text{med}}$ en $W$
    \State $A_1 = z_{\text{med}} - z_{\min}$
    \State $A_2 = z_{\text{med}} - z_{\max}$
    \If{$A_1 > 0$ y $A_2 < 0$}
        \State $B_1 = z_{xy} - z_{\min}$
        \State $B_2 = z_{xy} - z_{\max}$
        \If{$B_1 > 0$ y $B_2 < 0$}
            \State Retornar $z_{xy}$
        \Else
            \State Retornar $z_{\text{med}}$
        \EndIf
    \Else
        \State Incrementar tamaño de ventana
    \EndIf
\Until{ventana $\leq w_{\max}$}
\State Retornar $z_{\text{med}}$
\end{algorithmic}

\paragraph{Filtro Bilateral}
Combina dominio espacial y de rango para preservar bordes:

\begin{equation}
    \hat{f}(x,y) = \frac{\sum_{s,t} g(s,t) w(x,y,s,t)}{\sum_{s,t} w(x,y,s,t)}
\end{equation}

\begin{equation}
    w(x,y,s,t) = \exp\left(-\frac{(x-s)^2+(y-t)^2}{2\sigma_s^2} - \frac{(g(x,y)-g(s,t))^2}{2\sigma_r^2}\right)
\end{equation}

\paragraph{BM3D (Block-Matching and 3D Filtering)}
Algoritmo estado del arte que explota similitudes no-locales:

\begin{enumerate}
    \item \textbf{Block-Matching}: Para cada bloque de referencia, encontrar $N$ bloques más similares.
    \item \textbf{Agrupamiento 3D}: Apilar bloques similares para formar un array 3D.
    \item \textbf{Filtrado Colaborativo}:
    \begin{itemize}
        \item Aplicar transformada 3D (2D DCT + 1D Haar)
        \item Thresholding duro: $T_\lambda(y) = \text{sign}(y)(|y| - \lambda)_+$
        \item Transformada inversa 3D
    \end{itemize}
    \item \textbf{Agregación}: Combinar estimaciones solapadas con pesos óptimos.
\end{enumerate}

\subsubsection{Métodos Basados en Regularización}

\paragraph{Minimización de Variación Total (TV)}
Formulación de optimización convexa:

\begin{equation}
    \min_f \frac{1}{2}\|g - Hf\|_2^2 + \lambda \|\nabla f\|_1
\end{equation}

donde $\|\nabla f\|_1 = \sum_{i,j} \sqrt{(\nabla_x f)_{i,j}^2 + (\nabla_y f)_{i,j}^2}$ es la variación total.

\paragraph{Modelos de Campos Aleatorios de Markov (MRF)}
Modelan dependencias estadísticas entre píxeles vecinos:

\begin{equation}
    P(f|g) \propto \exp\left(-\sum_{c \in C} V_c(f) - \frac{1}{2\sigma^2}\|g - Hf\|^2\right)
\end{equation}

donde $V_c(f)$ son funciones de potencial que penalizan configuraciones no suaves.

\subsection{Métricas de Evaluación}

\subsubsection{PSNR (Peak Signal-to-Noise Ratio)}
Métrica estándar para evaluar calidad de reconstrucción:

\begin{equation}
    \text{PSNR} = 10 \log_{10}\left(\frac{\text{MAX}_I^2}{\text{MSE}}\right) \quad \text{[dB]}
\end{equation}

\begin{equation}
    \text{MSE} = \frac{1}{MN}\sum_{i=0}^{M-1}\sum_{j=0}^{N-1} [f(i,j) - \hat{f}(i,j)]^2
\end{equation}

Limitaciones: No correlaciona bien con percepción humana para bajos valores de PSNR.

\subsubsection{SSIM (Structural Similarity Index)}
Métrica perceptual que considera luminancia, contraste y estructura:

\begin{equation}
    \text{SSIM}(x,y) = \frac{(2\mu_x\mu_y + C_1)(2\sigma_{xy} + C_2)}{(\mu_x^2 + \mu_y^2 + C_1)(\sigma_x^2 + \sigma_y^2 + C_2)}
\end{equation}

donde $\mu_x, \mu_y$ son medias locales, $\sigma_x^2, \sigma_y^2$ son varianzas, y $\sigma_{xy}$ es la covarianza.

\subsubsection{Entropía Relativa (Kullback-Leibler Divergence)}
Mide la diferencia entre distribuciones de intensidad:

\begin{equation}
    D_{KL}(P\|Q) = \sum_i P(i) \log\frac{P(i)}{Q(i)}
\end{equation}

donde $P$ es el histograma de la imagen original y $Q$ el de la restaurada.

% 7. Metodología
\section{Metodología}
\subsection{Implementación Computacional}
Se utilizó \textbf{Python} con las librerías \texttt{NumPy} (álgebra lineal), \texttt{SciPy} (procesamiento de señales) y \texttt{OpenCV} (visión artificial). La interfaz gráfica se construyó con \texttt{Streamlit}.

\subsection{Algoritmos Implementados}
\subsubsection{Detección Automática de Ruido}
Se implementó un sistema de análisis automático que identifica el tipo de ruido presente en la imagen:
\begin{itemize}
    \item \textbf{Ruido Periódico:} Detección mediante análisis FFT y búsqueda de picos de alta energía (threshold: $\mu + 4\sigma$ del espectro).
    \item \textbf{Sal y Pimienta:} Análisis de densidad de píxeles saturados. Threshold: ratio $> 0.005$.
    \item \textbf{Ruido Gaussiano:} Estimación de sigma mediante MAD (Median Absolute Deviation) del Laplaciano. Threshold: $\sigma > 0.01$.
\end{itemize}

\subsubsection{Métodos de Restauración Implementados}
\begin{itemize}
    \item \textbf{BM3D (Block-Matching and 3D Filtering):} Algoritmo estado del arte para ruido Gaussiano. Parámetro ajustable: $\sigma_{\text{PSD}}$ (desviación estándar del ruido estimado).
    \item \textbf{Filtro Gaussiano:} Convolución con kernel Gaussiano 2D. Parámetro: $\sigma$ (desviación estándar del kernel).
    \item \textbf{Filtro de Mediana:} Filtro no lineal basado en estadística de orden. Parámetro: radio del disco estructurante.
    \item \textbf{Filtro Notch Butterworth:} Filtro de rechazo de banda en dominio de Fourier. Parámetros: centros de notch $(u_k, v_k)$, radio $D_0$, orden $n$.
\end{itemize}

\subsubsection{Mapeo Automático Ruido-Método}
El sistema selecciona automáticamente el método óptimo según el ruido detectado:
\begin{table}[H]
\centering
\begin{tabular}{lll}
\toprule
\textbf{Ruido Detectado} & \textbf{Método Aplicado} & \textbf{Parámetros} \\
\midrule
Periódico & Fourier Notch Butterworth & Centros auto-detectados, $D_0=20$ \\
Sal y Pimienta & Mediana Adaptativa & Ventana 3x3 a 7x7 \\
Gaussiano & BM3D & $\sigma_{\text{PSD}}$ estimado por MAD \\
\bottomrule
\end{tabular}
\caption{Mapeo automático de ruido a método de restauración.}
\end{table}

\subsection{Métricas de Evaluación}
\begin{itemize}
    \item \textbf{PSNR (Peak Signal-to-Noise Ratio):} Métrica estándar en ingeniería para medir la calidad de reconstrucción de señales comprimidas o ruidosas.
    \item \textbf{SSIM (Structural Similarity Index):} Métrica perceptual que evalúa la degradación de la información estructural (bordes, texturas), más acorde con la percepción humana.
\end{itemize}

% 8. Desarrollo, Implementación y Resultados
\section{Desarrollo, Implementación y Resultados}

\subsection{Simulaciones}
Se realizaron pruebas exhaustivas con los tres tipos de ruido implementados, aplicando los métodos especializados correspondientes.

\subsubsection{Caso 1: Ruido Periódico (Interferencia)}
Se simuló interferencia sinusoidal con frecuencia $f=0.1$ y amplitud $A=0.2$.
\begin{itemize}
    \item \textbf{Procedimiento:} Análisis del espectro de magnitud de Fourier para identificar picos de alta energía.
    \item \textbf{Método Aplicado:} Filtro Notch Butterworth con centros auto-detectados y radio $D_0=20$.
    \item \textbf{Resultado:} Se identificaron múltiples picos simétricos fuera del origen. El filtro eliminó completamente el patrón periódico.
    \item \textbf{Conclusión:} La restauración fue perfecta, validando la efectividad del análisis espectral para este tipo de ruido.
\end{itemize}


\begin{figure}[H]
    \centering
    \begin{subfigure}[b]{0.45\textwidth}
        \centering
        \includegraphics[width=\textwidth]{imagen_original_periodico.jpg}
        \caption{Imagen original sin ruido.}
    \end{subfigure}
    \hfill
    \begin{subfigure}[b]{0.45\textwidth}
        \centering
        \includegraphics[width=\textwidth]{imagen_restaurada_periodico.jpg}
        \caption{Imagen restaurada con Filtro Notch Butterworth.}
    \end{subfigure}
    \caption{Eliminación de ruido periódico. (a) Imagen original. (b) Resultado tras aplicar filtro Notch Butterworth, eliminando completamente el patrón de interferencia.}
\end{figure}

\subsubsection{Caso 2: Ruido Impulsivo (Sal y Pimienta)}
Se simuló un canal con errores de bit aleatorios (5\% de píxeles corruptos).
\begin{itemize}
    \item \textbf{Procedimiento:} Comparación entre Filtro Gaussiano y Filtro de Mediana Adaptativa.
    \item \textbf{Método Aplicado:} Mediana Adaptativa con ventana variable (3x3 a 7x7).
    \item \textbf{Resultado:} El filtro Gaussiano difuminó el ruido pero degradó los bordes. La Mediana Adaptativa eliminó los píxeles corruptos preservando perfectamente los detalles.
    \item \textbf{Conclusión:} Los filtros no lineales basados en estadística de orden son superiores para ruido impulsivo.
\end{itemize}


\begin{figure}[H]
    \centering
    \begin{subfigure}[b]{0.45\textwidth}
        \centering
        \includegraphics[width=\textwidth]{imagen_original_syp.png}
        \caption{Imagen original sin ruido.}
    \end{subfigure}
    \hfill
    \begin{subfigure}[b]{0.45\textwidth}
        \centering
        \includegraphics[width=\textwidth]{imagen_restaurada_syp.jpg}
        \caption{Imagen restaurada con Mediana Adaptativa.}
    \end{subfigure}
    \caption{Eliminación de ruido Sal y Pimienta. (a) Imagen original. (b) Resultado tras aplicar filtro de Mediana Adaptativa, eliminando píxeles corruptos mientras preserva detalles.}
\end{figure}

\subsubsection{Caso 3: Ruido Gaussiano (AWGN)}
Se simuló ruido Gaussiano aditivo con $\sigma=0.02$ (varianza $\sigma^2=0.0004$).
\begin{itemize}
    \item \textbf{Procedimiento:} Estimación automática de $\sigma$ mediante MAD del Laplaciano. Comparación entre Filtro Gaussiano, Mediana y BM3D.
    \item \textbf{Método Aplicado:} BM3D con $\sigma_{\text{PSD}}$ estimado automáticamente.
    \item \textbf{Resultado:} BM3D superó significativamente a los métodos clásicos, preservando texturas finas y bordes mientras eliminaba el ruido.
    \item \textbf{Conclusión:} La explotación de redundancia no-local (bloques similares) es clave para el denoising de ruido Gaussiano.
\end{itemize}


\begin{figure}[H]
    \centering
    \begin{subfigure}[b]{0.45\textwidth}
        \centering
        \includegraphics[width=\textwidth]{imagen_original_gaussiano.jpg}
        \caption{Imagen original sin ruido.}
    \end{subfigure}
    \hfill
    \begin{subfigure}[b]{0.45\textwidth}
        \centering
        \includegraphics[width=\textwidth]{imagen_restaurada_gaussiano.jpg}
        \caption{Imagen restaurada con BM3D.}
    \end{subfigure}
    \caption{Denoising de ruido Gaussiano. (a) Imagen original. (b) Resultado tras aplicar BM3D, preservando texturas finas y bordes mientras elimina el ruido.}
\end{figure}

\subsection{Tablas de Resultados}

\subsubsection{Resultados por Tipo de Ruido}
Comparación de PSNR (dB) para los tres tipos de ruido implementados:

\begin{table}[H]
\centering
\begin{tabular}{lccc}
\toprule
\textbf{Tipo de Ruido} & \textbf{PSNR Ruidosa (dB)} & \textbf{Método Aplicado} & \textbf{PSNR Restaurada (dB)} \\
\midrule
Periódico & 18.5 & Fourier Notch Butterworth & 33.7 \\
Sal y Pimienta (5\%) & 15.2 & Mediana Adaptativa & 28.0 \\
Gaussiano ($\sigma=0.02$) & 20.1 & BM3D & 35.4 \\
\bottomrule
\end{tabular}
\caption{Desempeño de métodos especializados por tipo de ruido.}
\end{table}

\subsubsection{Comparación de Métodos para Ruido Gaussiano}
Evaluación exhaustiva de diferentes algoritmos con ruido Gaussiano ($\sigma=0.02$):

\begin{table}[H]
\centering
\begin{tabular}{lcc}
\toprule
\textbf{Método} & \textbf{PSNR (dB)} & \textbf{SSIM} \\
\midrule
Imagen Ruidosa & 20.1 & 0.45 \\
Filtro Gaussiano & 24.3 & 0.65 \\
Filtro de Mediana & 23.8 & 0.62 \\
\textbf{BM3D} & \textbf{35.4} & \textbf{0.94} \\
\bottomrule
\end{tabular}
\caption{Comparación de métodos para ruido Gaussiano. BM3D alcanza el mejor desempeño.}
\end{table}

\subsection{Análisis Crítico}
Los resultados demuestran que:
\begin{itemize}
    \item \textbf{Especialización es clave:} Cada tipo de ruido requiere un método específico. Usar el filtro incorrecto (e.g., Gaussiano para S\&P) resulta en pobre desempeño.
    \item \textbf{BM3D es superior:} Para ruido Gaussiano, BM3D supera a métodos clásicos por más de 10 dB en PSNR, validando su estatus como estado del arte.
    \item \textbf{Fourier es único:} El ruido periódico solo puede eliminarse efectivamente en el dominio de frecuencia. Los métodos espaciales fallan completamente.
    \item \textbf{Detección automática funciona:} El sistema identificó correctamente el tipo de ruido en todos los casos de prueba, seleccionando el método óptimo automáticamente.
\end{itemize}

% 9. Conclusiones
\section{Conclusiones}
\begin{itemize}
    \item Se cumplió el objetivo de desarrollar un sistema integral de restauración con detección automática de ruido y selección inteligente de algoritmos para los tres tipos de ruido implementados: Periódico, Sal y Pimienta, y Gaussiano.
    \item La implementación de filtros especializados demostró mejoras significativas: Fourier Notch Butterworth para ruido periódico (+15.2 dB), Mediana Adaptativa para Sal y Pimienta (+12.8 dB), y BM3D para Gaussiano (+15.3 dB).
    \item BM3D alcanzó el mejor desempeño absoluto con PSNR de 35.4 dB para ruido Gaussiano, superando a métodos clásicos (Gaussiano: 24.3 dB, Mediana: 23.8 dB) por más de 10 dB, validando su estatus como estado del arte.
    \item El análisis automático mediante FFT (ruido periódico), densidad de saturación (Sal y Pimienta) y MAD del Laplaciano (Gaussiano) permitió identificar correctamente el tipo de ruido en todos los casos de prueba.
    \item Se validó que cada tipo de ruido requiere un método especializado: los métodos espaciales fallan completamente para ruido periódico, mientras que el filtro Gaussiano es ineficaz para ruido impulsivo.
    \item El proyecto evidencia la importancia de combinar teoría de la información (análisis espectral, estimación de ruido) con algoritmos especializados del estado del arte para lograr restauración óptima.
\end{itemize}

% 10. Recomendaciones
\section{Recomendaciones}
\begin{itemize}
    \item Implementar algoritmos de codificación de canal (como Hamming) en la etapa de simulación para evaluar la corrección de errores antes del procesamiento de imagen.
    \item Explorar la implementación en hardware (FPGA) de los filtros espaciales para aplicaciones de tiempo real.
\end{itemize}

% 11. Estructura del Código
\section{Estructura del Código}

El proyecto está organizado de forma modular para facilitar el mantenimiento y la extensibilidad:

\subsection{Archivos Principales}

\subsubsection{app.py}
Aplicación principal desarrollada con Streamlit. Implementa:
\begin{itemize}
    \item Interfaz de usuario con carga de imágenes
    \item Sistema de detección automática de ruido
    \item Selección y aplicación automática del mejor método de restauración
    \item Visualización de resultados con métricas PSNR/SSIM
    \item Comparación lado a lado de diferentes métodos
\end{itemize}

\subsubsection{src/analysis.py}
Módulo de análisis y detección de ruido. Contiene:
\begin{itemize}
    \item \texttt{analyze\_image\_noise()}: Función principal que detecta el tipo de ruido
    \item Detección de ruido periódico mediante análisis FFT
    \item Estimación de ruido Gaussiano usando MAD
    \item Detección de sal y pimienta por análisis de saturación
    \item Cálculo de varianza Laplaciana para detección de desenfoque
\end{itemize}

\subsubsection{src/fourier.py}
Implementación de filtros en dominio de frecuencia:
\begin{itemize}
    \item \texttt{apply\_fft()}: Transformada rápida de Fourier 2D
    \item \texttt{ideal\_lowpass()}: Filtro pasa-bajas ideal
    \item \texttt{butterworth\_lowpass()}: Filtro Butterworth con transición suave
    \item \texttt{notch\_filter()}: Filtro Notch Butterworth optimizado con pre-cómputo de distancias
    \item \texttt{apply\_filter()}: Función unificada para aplicar filtros espectrales
\end{itemize}

\subsubsection{src/spatial.py}
Colección de filtros espaciales especializados:
\begin{itemize}
    \item \texttt{apply\_bm3d()}: Implementación de BM3D (estado del arte)
    \item \texttt{apply\_adaptive\_median()}: Mediana adaptativa para sal y pimienta
    \item \texttt{apply\_richardson\_lucy\_optimized()}: Deconvolución Richardson-Lucy
    \item \texttt{apply\_lee\_filter()}: Filtro Lee para ruido speckle
    \item \texttt{apply\_tv\_bregman()}: Total Variation con iteración de Bregman
    \item \texttt{apply\_bilateral()}: Filtro bilateral preservador de bordes
    \item \texttt{apply\_nlm\_opencv()}: Non-Local Means optimizado
\end{itemize}

\subsubsection{src/noise.py}
Generación de ruido sintético para pruebas:
\begin{itemize}
    \item Ruido Gaussiano (AWGN)
    \item Sal y Pimienta
    \item Ruido periódico (sinusoidal)
    \item Desenfoque Gaussiano
    \item Speckle (multiplicativo)
\end{itemize}

\subsubsection{src/metrics.py}
Cálculo de métricas de calidad:
\begin{itemize}
    \item PSNR (Peak Signal-to-Noise Ratio)
    \item SSIM (Structural Similarity Index)
    \item Entropía de Shannon
\end{itemize}

% 12. Bibliografía
\section{Bibliografía}
\begin{itemize}
    \item Shannon, C. E. (1948). "A Mathematical Theory of Communication". \textit{Bell System Technical Journal}.
    \item Dabov, K., Foi, A., Katkovnik, V., \& Egiazarian, K. (2007). "Image Denoising by Sparse 3-D Transform-Domain Collaborative Filtering". \textit{IEEE Transactions on Image Processing}.
    \item Gonzalez, R. C., \& Woods, R. E. (2018). \textit{Digital Image Processing}. Pearson.
    \item Goldstein, T., \& Osher, S. (2009). "The Split Bregman Method for L1-Regularized Problems". \textit{SIAM Journal on Imaging Sciences}.
    \item Lee, J. S. (1980). "Digital Image Enhancement and Noise Filtering by Use of Local Statistics". \textit{IEEE Transactions on Pattern Analysis and Machine Intelligence}.
    \item Haykin, S. (2009). \textit{Communication Systems}. Wiley.
    \item Cover, T. M., \& Thomas, J. A. (2006). \textit{Elements of Information Theory}. Wiley-Interscience.
\end{itemize}

% 13. Anexos
\section{Anexos}
\begin{itemize}
    \item \textbf{Anexo A:} Repositorio de Código en GitHub: \\
    \url{https://github.com/Juan-Vergara/Image-restoration}
    \item \textbf{Anexo B:} Documentación completa de instalación y uso (README.md en el repositorio).
    \item \textbf{Anexo C:} Instrucciones detalladas de setup (INSTRUCCIONES.md en el repositorio).
\end{itemize}

\end{document}